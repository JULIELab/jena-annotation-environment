\documentclass[DIV12,english,11pt,halfparskip]{scrartcl}
\usepackage[pdftex]{graphicx}
\usepackage[latin1]{inputenc}
\usepackage[headsepline]{scrpage2}
\usepackage{amsmath}
\usepackage{amssymb}
\usepackage{graphicx}
\usepackage{xcolor}
\usepackage{natbib}
%\usepackage{subfigure}
\usepackage[pdfpagemode=None,%
%linkbordercolor = 0 0 0,%
linkcolor = red,%
%anchorcolor = 0 0 0,%
citecolor = blue,%
%filecolor = 0 0 0,%
%5menucolor = 0 0 0,%
%pagecolor = 0 0 0,%
urlcolor = violet,%
colorlinks,%
plainpages=false, pdfpagelabels,%
%backref,%
pdftitle={JANE - Demo HowTo},%
pdfsubject={},%
pdfauthor={Katrin Tomanek},%
pdfkeywords={Jena Annotation Environment, JANE},%
pdfcreator={JULIE Lab, University of Jena},%
pdfproducer={pdfeTeX},%
]{hyperref}
\pagestyle{scrheadings}
\automark{section}

\title{JANE Demo}

\author{\normalsize Katrin Tomanek\\
  \normalsize  Jena University Language \& Information Engineering (JULIE) Lab\\
  \normalsize F\"urstengraben 30 \\
  \normalsize D-07743 Jena, Germany\\
  {\normalsize \tt tomanek@coling-uni-jena.de} } \date{}

\begin{document}
\maketitle
\newpage
\tableofcontents
\newpage

\section{About the Demo}
As installing the complete annotation environment JANE is a bit
tricky, the demo version allows you to explore the annotation part of
JANE with minimal installation effort (see section
\ref{sec:installation}).

In the demo version, you have only access to the annotation component.
In the demo version, both the annotation repository and the
administration component are hosted at a server at the JULIE Lab and
cannot be accessed directly by you. The active learning selection can
be used by you from the annotation component, however, it is executed
at a server at the JULIE Lab, as well.

\section{Demo Annotation Projects}
I preconfigured a small collection of annotation projects for you to
play with:

\begin{description}
\item[Default projects] there are two default projects, one for the
  bio-medical domain and one for the news-domain. You might want to
  play around with these projects a little bit to get accustomed to
  using MMAX2 for annotations.

\item[AL projects of bio-medical domain] there are three projects: a)
  \emph{generic} (you have one entity label which you can assign to
  whatever you consider as the entity of interest (e.g., proteins). b)
  \emph{cell types} (also there is only one label but you should
  assign it only to cells, as the initial sentence selection for AL is
  tuned a little bit to this entity type) and c) \emph{organism}
  (where you can assign four different organism types -- human, mouse,
  rat, and virus).

  These AL projects select new sentences from a large pool of abstract
  (about 130,000) downloaded from Medline.

\item[AL project for the news domain] there is one project for the
  news paper domain where you can assign the entitiy types
  \emph{person}, \emph{location}, and \emph{organization}.

  This AL project has a pool of about 16,000 Reuters news messages
  (from 1997) from which it selects the sentences for annotation.

\end{description}


Please note that for the AL projects, the initial selection of
sentences to be annotated was done in a very generic way to allow
flexible use with different annotation guidelines. This, however, will
not result in the most performant AL selection, which would be
achieved when more carefully selection the initial sentence selection.
When planning to do real (productive) annotations with AL one should
consider this!

To get a personalized demo user, please contact me. Further, if you
want your annotations to be exported (at the moment we offer a format
similar to the IOB format frequently used in the bio text mining
community), also contact me.

Please refer to the annotation GUI HowTo for detailed information on
how to use the GUI and the active learning selection.


\section{Installation}
\label{sec:installation}
To install the JANE demo, i.e. the annotatio GUI of JANE, on your PC,
just download the demo from \url{http://www.julielab.de} and unpack
the file with the following command:
\begin{verbatim}
tar xzvf JANE-demo-1.0.tgz
\end{verbatim}

This will create a new directory called \url{JANE-demo-1.0}. Go to
this directory and start the installation script:
\begin{verbatim}
install_JANE-demo.sh
\end{verbatim}

You will be asked for the following configuration parameters:

\begin{description}
\item [Base directory of JANE] enter the complete path to your JANE
  installation. Note: finish it with a slash (``/'')

  Example: if I unpack JANE in my home directory (\url{/home/tomanek})
  I would have to write: \url{/home/tomanek/JANE-demo-1.0/}

\item[Directory for temporary files] enter the complete path to a
  directory (you need write access) where JANE can store
  temporary files. Just press enter to accept the defaul (\url{/tmp}).

\item[Directory with MMAX libraries] MMAX2 is the annotation editor
  used by JANE. You can use the MMAX2 libraries provided with your JANE
  demo. These are contained in the subdirectory \url{libs/MMAX2_1.1}.
  You have to enter the full path to the MMAX2 libraries.

  Example: I might enter
  \url{/home/tomanek/JANE-demo-1.0/lib/MMAX2_1.1}

\end{description}

Now, you demo version of JANE is installed\footnote{Note: please do
  not change any other settings as otherwise the demo version will not
  run properly.}. To start the annotation GUI, go to the subdirectory
\url{bin} and execute the script \url{runAnnoClient.sh}


\section{Copyright and License}

This software is Copyright (C) 2007 Jena University Language \&
Information Engineering Lab (Friedrich-Schiller University Jena,
Germany). It can be used free of charge for academic non-commercial,
non-profit research purposes only. If you want to use JANE other than
in the demo version you have to fill out and send us the license
agreement
(\url{http://www.julielab.de/images/stories/licence-julie-jane.pdf}).

JANE is provided on an ``as is'' basis, without any warranties or
conditions of any kind. Each recipient is solely responsible for
determining the appropriateness of using JANE.

Further, please note that JANE-1.0 is still a beta version!


\end{document}
